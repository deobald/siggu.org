\documentclass{article}

\usepackage{polyglossia}
\setdefaultlanguage{english}
\setotherlanguage{sanskrit}
\usepackage{fontspec}
\newfontfamily\devanagarifont[Scale=MatchUppercase]{Kohinoor Devanagari}

\usepackage{graphicx}
\usepackage{subcaption}
\usepackage{float}
\usepackage[bottom]{footmisc}
\usepackage{enumitem}
\usepackage{hyperref}

\setlength{\parindent}{1.3em}
\setlength{\parskip}{0.7em}
\setlist{noitemsep}

\begin{document}

Dear Vipassana Centre Workers,

Over the past three years, I have been fortunate enough to meditate vipassana courses
in your centres at times and to serve with you at other times. When we serve together
we are doing simple but important work: chopping vegetables, making chappatis,
cleaning vessels, cleaning the rooms, washing linens, constructing new buildings, and
painting walls. When I speak to people about this work, it is hard for me to explain
why it is so important. I will try to explain in the form of a story.

\begin{center}
\line(1,0){250}
\end{center}

I came to India in 2012. I was rich but very unhappy. I was so rich I could fly
anywhere I wanted, buy any car I wanted, eat in any restaurant... but it didn't feel
like it was enough. I always wanted more.

At that time I was a drunkard. I would smoke cigarettes and
use drugs. I would sleep strange hours. I was an unreliable employee --- I would come
late to work and often I would not even come to work because of laziness. I would eat
unhealthy food and I rarely exercised. Because I was so unhealthy, I would get sick
very often.

I was fat --- in my body and in my mind.

Because my mind was unhealthy, I would make the wrong decisions. Around my friends
and coworkers I would use hurtful words and I would shout. I was angry all the
time. I became so angry some days it hurt to breathe.

I became very careless. Sometimes I was even so careless I would physically hurt
people. One time I was riding my bicycle very fast and hit an old couple on a
scooter. The old lady went flying off the back of the scooter and hurt her hand. I
was even more careless with my friends and family... I would take them for granted
and never showed that I really appreciated them. I was always expecting that someone
else must do something for me.

Finally, my life reached a breaking point. I could not sleep at all. I was smoking
one cigarette after another. I was drinking and smoking marijuana every day to dull
my thoughts and emotions. It felt as if I had gone crazy.

And then everything changed. My friend suggested I take a 10-day Vipassana
course. Let me be clear: There was no magic and my problems did not disappear
overnight. Vipassana is not a miracle cure --- I had to work very hard during those
ten days. But even within that first course I truly understood how I was hurting myself
and hurting everyone around me. After that, nothing was ever the same again.

It has been seven years since my first 10-day Vipassana course. I ran my first
half-marathon this past July. I do not drink, smoke, or do drugs any more. I am the
healthiest I have ever been in my life. I sleep peacefully. Now I speak gently to
family and friends and I try to show them the love they deserve. Now I spend my energy
trying to improve the world around me.

Vipassana saved my life.

\begin{center}
\line(1,0){250}
\end{center}

This is why our work is so important. This is why your work is so important. There
are many people like me --- confused, angry, and full of greed.

You are doing so much more than feeding people, cleaning for people,
or building a place for people to stay. Cooking for people, cleaning for people, and
building for people --- these are all very important jobs, even on their own. But
with every chappati you cook, with every vessel you clean, with every brick
you lay down for a new building you are giving those same people a place to meditate
Vipassana. This gives each and every person who comes to the Vipassana centre an
opportunity to change the way I have changed. And once they change, they too can
focus their energy to improve the world around them, as you do.

It is for this reason that your work at the Vipassana Centre is so important. It is
for this reason that I am so grateful for the work that you do.

Be Happy,

-steven

\end{document}
