\documentclass{article}

\usepackage{polyglossia}
\setdefaultlanguage{hindi}
\setotherlanguage{sanskrit}
\usepackage{fontspec}
\newfontfamily\devanagarifont[Scale=MatchUppercase]{Kohinoor Devanagari}

\usepackage{graphicx}
\usepackage{subcaption}
\usepackage{float}
\usepackage[bottom]{footmisc}
\usepackage{enumitem}
\usepackage{hyperref}

\setlength{\parindent}{1.3em}
\setlength{\parskip}{0.7em}
\setlist{noitemsep}

\begin{document}

प्रिय विपाशयना केंद्र सेवकों,

पिछले तीन सालों में मैं बहुत भाग्यशाली रहा कि आपके विपाशयना केंद्र में मुझे कई बार ध्यान करने का मौका मिला और साथ
ही कई बार सेवा करने का भी मौका मिला। साथ सेवा करते हुए, हम सामान्य पर महत्यपूर्ण कार्य करते हैं: सब्ज़ियाँ काटना, रोटियाँ
बनाना, बर्तन धोना, कमरे साफ़ करना, कपड़े धोना, नई इमारतें बनाना और दीवारें रंगना। जब कभी मैं
लोगों से इन कामों के बारे में बात करता हूँ, उन्हें ये समझाना बहुत मुश्किल होता है कि ये काम इतने ज़रूरी क्यों हैं।
ंमैं एक कहानी के माध्यम से यह समझाने की कोशिश करूँगा।


\begin{center}
\line(1,0){250}
\end{center}

२०१२ में मैं भारत आया। मैं अमीर था पर बहुत दुखी। मैं इतना अमीर था कि मैं दुनिया में कहीं भी जा सकता
था, कोई भी कार ख़रीद सकता था, किसी भी रेस्तराँ में खा सकता था ... पर मुझे कभी
नहीं लगता था कि ये काफ़ी है। मैं हमेशा ज़्यादा की चाह रखता था।

उस समय मैं बहुत शराब पीता था। मैं बहुत धूम्रपान करता था और नशीली दवाएं लेता था। मैं अजीब घंटो पर सोता
था। मैं एक ग़ैरज़िम्मेदार कर्मचारी था \textemdash मैं काम पर देर से पहुँचता था और अक्सर अपने आलस के कारण
काम पर जाता ही नहीं था। मैं बहुत अस्वस्थकर खाना खाता था और कभी कभार ही कसरत करता था। इसलिए मैं
बहुत बीमार भी पड़ा करता था।

मैं शारीरिक और मनिसिक रूप से टूट गया था।

क्योंकि मेरा मन इतना बीमार था, मैं अक्सर ग़लत निर्णय लेता था। मैं अपने दोस्तों और सहकर्मचारियों को बहुत
दुखदाई बातें कहता था और उनपर चिल्लाता भी था। मैं हर समय ग़ुस्सा रहेता था। मैं इतना ग़ुस्सा रहेता था
कि साँस लेने में भी दर्द होता था।

मैं बहुत लापरवाह था। कभी कभी मैं इतना लापरवाह होता था कि मैं ख़ुद को और दूसरों को शारीरिक रूप से
चोट पहुँचता था। एक बार मैं बहुत तेज़ी से अपनी साइकल चला रहा था और मैंने एक बूढ़े जोड़े को टक्कर मार दी। वो बूढ़ी औरत स्कूटेर के पीछे से उछल कर गिर पड़ीं और उनके हाथ में चोट लग गयी।

मैं इससे भी ज़्यादा लापरवाह अपने दोस्तों और
परिवार के साथ था। मैं हमेशा दूसरों से उम्मीद करता था कि वह मेरे लिए कुछ करें।

आख़िरकार, मैं अपने जीवन के बिखराव के बिंदु तक पहुँच गया। मैं बिलकुल भी नहीं सो पाता था। मैं एक के
बाद एक सिगरेट पी रहा था। मैं रोज़ाना शराब पीता था और गांजा लेता था ताकी मेरी भावनाओं और विचारों
को दबा सकू। मुझे एसा लगता था की मैं पागल हो चुका हूँ।

और फिर सबकुछ बदल गया। मेरे एक दोस्त ने मुझे एक दस दिन के विपाशयना शिविर में जाने की राय दी। मुझे स्पष्ट
रूप से कहने दीजिए: शिविर में कोई जादू नहीं हुआ और ना ही मेरे परेशनियाँ एक दिन में ग़ायब हो गयीं।
विपाशयना कोई चमत्कारी इलाज नहीं है \textemdash मुझे उन दस दिनो में बहुत मेहनत करनी पड़ी। लेकिन उस पहले
शिविर के दौरान ही मुझे सच में समझ में आ गया कि मैं किस तरह ख़ुद को और मेरे आसपास के सभी लोगों को चोट
पहुँचा रहा था। उसके बाद कुछ भी पहले जैसा नहीं रहा।

उस पहले विपाशयना शिविर से लेकर अब तक सात साल हो गये हैं। पिछली जुलाई में मैंने मेरा पहला आधा मैराथन
दौड़ा। मैं अब शराब और सिगरेट नहीं पीता और ना ही नशीली दवाएं लेता हूँ। मैं अपने जीवन में इतना स्वस्थ पहले
कभी नहीं रहा। अब मैं शांति से सोता हूँ। मैं अपने परिवार और दोस्तों से विनम्रता से बात करता हूँ और हमेशा
कोशिश करता हूँ कि उन्हें वो प्यार और सम्मान दे सकूँ जिसके वो हमेशा से हक़दार थे। अब मैं अपनी ऊर्जा अपने
आसपास की दुनिया को बहेतर बनाने में लगता हूँ।

विपाशयना ने मेरे इस जीवन को बचाया है।

\begin{center}
\line(1,0){250}
\end{center}

इसलिए हमारा और आपका ये काम इतना महत्यपूर्ण है। यहाँ मेरे जैसे बहुत सारे लोग हैं \textemdash उलझन में, नाराज़, और लालच से भरे हुये।

TODO: revisit
ignorance (confusion)
craving (greed)
aversion (anger)

आप लोग सिर्फ़ लोगों को खाना खिलाने, उनके लिए सफ़ाई करने, या उनके रहेने के लिए जगहें बनाने से बहुत
ज़्यादा करते हैं। ये सभी अपने आप में बहुत ज़रूरी काम हैं पर हर वो चपाती जो आप बनाते हैं, हर वो बर्तन जो आप धोते हैं, हर वो ईंट जो आप एक नयी इमारत में लगाते हैं, उसके साथ आप मेरे जैसे लोगों को ध्यान करने के लिए एक जगह देते हैं। TODO: revisit this sentence.
ये हर उस व्यक्ति को बदलाव का एक मौका देता है जो इन विपाशयना शिविर में
आता है, जैसा मुझे मिला था। और जब वो लोग बदलते हैं तो वो भी अपनी ऊर्जा को इस दुनिया को बेहतर
बनाने में लगा सकते हैं, जैसा आप लोग करते हैं।

यही कारण है कि जो काम आप विपाशयना केंद्र में करते हैं, वो इतना महत्यपूर्ण है।

इसी कारण से मैं हमेशा आप लोगों के काम के प्रति आभारी रहूँगा।

मंगल हो,

-स्टीवन

\end{document}
